%================================================================
% H1 paper
%================================================================
\RequirePackage{lineno}
\documentclass[12pt]{article}


\usepackage{epsfig}
\usepackage{hhline}
\usepackage{amsmath}
\usepackage{amssymb}
\usepackage{color}
\usepackage{xspace}
\usepackage{xfrac}
\usepackage{paralist} % compactitem
\usepackage{caption}
%\captionsetup[figure]{font=footnotesize,labelfont=footnotesize}
\captionsetup{font=small,labelfont=small}

\usepackage{bm} % 'bold' math symbols (better use \usepackage{newtxtext,newtxmath}, if available on latex distribution)
\usepackage{acronym}

%%%%%%%%%%%%% Comment the next two lines to remove the line numbering
\usepackage[]{lineno}
\linenumbers
%%%%%%%%%%%%%%

\usepackage{hyperref} % has to be last package loaded
\hypersetup{colorlinks=true, urlcolor=blue}
\usepackage{cite} % DB: enable also clickable references (must be loaded after hyperref)
\hypersetup{
  colorlinks,
  citecolor=blue,
  linkcolor=red,
  urlcolor=blue
  }


%%%%%%%%%%%%%% H1 preliminary
%\renewcommand{\topfraction}{1.0}
%\renewcommand{\bottomfraction}{1.0}
%\renewcommand{\textfraction}{0.0}
%\renewcommand{\arraystretch}{1.3} % make lines a bit larger for tables 
%%%%%%%%%%%%%%

%%%%%%%%%%%%%% H1 paper layout %%%%%%%%%%%%%%
\renewcommand{\topfraction}{1.0}
\renewcommand{\bottomfraction}{1.0}
\renewcommand{\textfraction}{0.0}
\renewcommand{\arraystretch}{1.25} % make lines a bit larger for tables
%\newlength{\dinwidth}
%\newlength{\dinmargin}
%\setlength{\dinwidth}{21.0cm}
%\textheight23.5cm \textwidth16.0cm
%\setlength{\dinmargin}{\dinwidth}
%\setlength{\unitlength}{1mm}
%\addtolength{\dinmargin}{-\textwidth}
%\setlength{\dinmargin}{0.5\dinmargin}
%\oddsidemargin -1.0in
%\addtolength{\oddsidemargin}{\dinmargin}
%\setlength{\evensidemargin}{\oddsidemargin}
%\setlength{\marginparwidth}{0.9\dinmargin}
\marginparsep 8pt \marginparpush 5pt
\topmargin -42pt
\headheight 12pt
\headsep 30pt \footskip 24pt
\parskip 3mm plus 2mm minus 2mm

% do not indent first line of paragraph!
%\setlength{\parindent}{0pt}
\usepackage{parskip}
%%%%%%%%%%%%%%%%%%%%%%%%%%%%%%%%%%%%%%%%%%%%%




%%% contains utf-8, see: http://inspirehep.net/info/faq/general#utf8
%%% add \usepackage[utf8]{inputenc} to your latex preamble
\usepackage[utf8]{inputenc}
%\bibliographystyle{plain}%Choose a bibliograhpic style          
\bibliographystyle{utphys}%Choose a bibliograhpic style          

\newlength{\dinwidth}
\newlength{\dinmargin}
\setlength{\dinwidth}{21.0cm}
\textheight23.5cm \textwidth16.0cm
\setlength{\dinmargin}{\dinwidth}
\setlength{\unitlength}{1mm}
\addtolength{\dinmargin}{-\textwidth}
\setlength{\dinmargin}{0.5\dinmargin}
\oddsidemargin -1.0in
\addtolength{\oddsidemargin}{\dinmargin}
\setlength{\evensidemargin}{\oddsidemargin}
\setlength{\marginparwidth}{0.9\dinmargin}
\marginparsep 8pt \marginparpush 5pt
\topmargin -42pt
\headheight 12pt
\headsep 30pt \footskip 24pt
\parskip 3mm plus 2mm minus 2mm
\setlength{\parindent}{0.0cm} 
\newcommand{\picob}{\mbox{{\rm ~pb}}}
\newcommand{\QQ}  {\mbox{${Q^2}$}}

\newcommand{\NNLOJET}{NNLO\protect\scalebox{0.8}{JET}\xspace}

%===============================title page=============================

% Some useful tex commands
%
%\def\GeV{\hbox{$\;\hbox{\rm GeV}$}}
%\def\MeV{\hbox{$\;\hbox{\rm MeV}$}}
%\def\TeV{\hbox{$\;\hbox{\rm TeV}$}}

\newcommand{\pb}{\rm pb}
\newcommand{\cm}{\rm cm}
\newcommand{\hdick}{\noalign{\hrule height1.4pt}}

\begin{document}
\pagestyle{empty}

\newcommand{\GeVsq}{\ensuremath{\mathrm{GeV}^2} }
\newcommand{\GeV}{\ensuremath{\mathrm{GeV}} }
\newcommand{\pt}{\ensuremath{P_{T}}}
\newcommand{\PP}{\ensuremath{\mathcal{P}}}
%\newcommand{\ptAvg}{\ensuremath{\langle P_T^{jet}\rangle}}
\newcommand{\Qsq}{\ensuremath{Q^{2}}}
%% unfolding
\newcommand{\chisq}{\ensuremath{\chi^{2}}}
\newcommand{\chisqA}{\ensuremath{\chi_{\rm A}^{2}}}
\newcommand{\chisqL}{\ensuremath{\chi_{\rm L}^{2}}}
\newcommand{\ndf}{\ensuremath{n_{\rm dof}}}
\newcommand{\A}{\ensuremath{\bm{A}}}
\newcommand{\V}{\ensuremath{\bm{V}}}
\newcommand{\B}{\ensuremath{\bm{B}}}
\newcommand{\J}{\ensuremath{\bm{J}}}
\newcommand{\N}{\ensuremath{\bm{N}}}
\newcommand{\LL}{\ensuremath{\bm{L}}}

\newcommand{\etajet}{\ensuremath{\eta_{\rm lab}^{\rm jet}}}
\newcommand{\ptjet}{\ensuremath{P_{\rm T}^{\rm jet}}}
\newcommand{\meanpt}{\ensuremath{\langle P_{\rm T} \rangle}}
\newcommand{\etalab}{\ensuremath{\eta_{\rm lab}^{\rm jet}}}
\newcommand{\Mjj}{\ensuremath{m_{12}}}
\newcommand{\meanptdi}{\ensuremath{\langle P_{\mathrm{T}} \rangle_{2}}\xspace}
\newcommand{\meanpttri}{\ensuremath{\langle P_{\mathrm{T}} \rangle_{3}}\xspace}
\newcommand{\mz}{\ensuremath{m_{\rm Z}}\xspace}
\newcommand{\as}{\ensuremath{\alpha_{\rm s}}\xspace}
\newcommand{\asmz}{\ensuremath{\as(\mz)}\xspace}
\newcommand{\etal}{{\it{et al.}}}
\newcommand{\mur}{\ensuremath{\mu_{R}}\xspace}
\newcommand{\muf}{\ensuremath{\mu_{F}}\xspace}
\newcommand{\murf}{\ensuremath{\mu_{R/F}}\xspace}
\newcommand{\asmur}{\ensuremath{\alpha_{\rm s}(\mur)}\xspace}
\newcommand{\chad}{\ensuremath{c_{\rm had}}\xspace}
\newcommand{\ord}{\ensuremath{\mathcal{O}}\xspace}


% % % Journal macro
% % \def\Journal#1#2#3#4{{#1}~{\bf #2} (#3) #4}
% % %\def\NCA{\em Nuovo Cimento}
% % %\def\NIM{\em Nucl. Instrum. Methods}
% % %\def\NIMA{{\em Nucl. Instrum. Methods} {\bf A}}
% % %\def\NPB{{\em Nucl. Phys.}   {\bf B}}
% % %\def\PLB{{\em Phys. Lett.}   {\bf B}}
% % %\def\PRL{\em Phys. Rev. Lett.}
% % %\def\PRD{{\em Phys. Rev.}    {\bf D}}
% % %\def\ZPC{{\em Z. Phys.}      {\bf C}}
% % %\def\EJC{{\em Eur. Phys. J.} {\bf C}}
% % %\def\CPC{\em Comp. Phys. Commun.}
% % %
% % \def\NPB{Nucl. Phys.~}
% % \def\PRL{Phys. Rev. Lett.~}
% % \def\EPJC{Eur. Phys. J.~}
% % \def\PLB{Phys. Lett.~}
% % \def\NIM{Nucl. Instrum. Meth.~}
% % \def\PRD{Phys. Rev.~}
% % \def\JHEP{JHEP~}
% % \def\PROC{Conf. Proc.~}
% % \def\CPC{Comp. Phys. Commun.~}
% % 


\newcommand{\sHat}{\ensuremath{\hat\sigma}}
\newcommand{\asA}{\ensuremath{\alpha_s^A}}
\newcommand{\asB}{\ensuremath{\alpha_s^B}}
\newcommand{\asC}{\ensuremath{\alpha_s^C}}
\newcommand{\asD}{\ensuremath{\alpha_s^D}}


%%%%%%%%%%%%%%%%%%%%%%%%%%%%%%%%%%%%%%% title page %%%%%%%%%%%%%%%%%%%%%%%%%%%%%%%%%%%%%%%%
\vspace{1cm}
\begin{center}
\begin{Large}

{\bf     \boldmath
  PDFs and $\alpha_s(m_Z)$
}
\end{Large}
\end{center}
\vspace{0.5cm}

\section{Cross section definition}
In $ep$ the QCD cross section is calculated as
\begin{align}
  \sigma^{(0)} &= \sHat(\muf,\asA) \otimes f(\muf).
\label{eq:sig0}
\end{align}
with $\sHat(\muf,\asA)$ being the perturbatively calculable hard coefficients
and $f(\muf)$ being the PDFs at the factorisation scale \muf.
The superscript `$A$' to \as\ denotes that the value of \asmz\ may become different than in other parts of the calculation.

Following notes by Valerio, the DGLAP evolution can be written
in terms of the evolution operator $\Gamma$. 
This allows to obtain a PDF at any other scale \muf\ obtained from an initial 1-dimensional $x$-dependent PDF $f(\mu_0)$ as
\begin{equation}
  f(\muf,\asD) = \Gamma(\muf,\mu_0,\asD) \otimes f
\end{equation}
The PDF $f$ was determined using a value of \asD. One thus may write: $f = f(\mu_0,\asD)$.

The DGLAP evolution operator is defined (following Valerios notes) as:
\begin{align}
  \Gamma(\mu_0,\muf,\asD)  = \exp\left[ \int_{\ln\mu^2_0}^{\ln\muf^2} P(\alpha_s(\mu^\prime) ) d\ln\mu^{\prime2} \right]
\end{align}



\section{Current fitting approach using \boldmath\muf-shift}
In the currently applied fitting approach, using the \muf-shift formalism, the cross section
prediction is obtained as:
\begin{align}
  \sigma^{(I)} &= \sHat(\muf,\asA) \otimes \gamma(\muf,\asA,\asD) \otimes f(\muf,\asD).
\label{eq:current}
\end{align}
The `\muf-shift' operator is given by (following notes by Valerio):
\begin{align}
  \gamma(\muf,\asA,\asD) = \exp\left[\int_{\asD}^{\asA} \frac{P(\alpha_s^\prime)}{\beta(\alpha_s^\prime) } d\alpha_s^\prime \right]. 
\end{align}
The cross section in eq.~\ref{eq:current} implies that the PDF is known for a given
value of \asD\ at all scales \muf, and herewith we translate it to the same value
\asA\ as used for the matrix elements.


However, the scale-dependence of the employed PDF, is not determined from data, but taken from
the DGLAP evolution. 
The PDF fitting groups determine only a PDF at $\mu_0$ and it is evolved to \muf. Thus 
equation~\ref{eq:current} can be rewritten as,
\begin{align}
  \sigma^{(I)}  &= \sHat(\muf,\asA) \otimes \gamma(\muf,\asA,\asD) \otimes \Gamma(\mu_0,\muf,\asD) \otimes f,
\end{align}
where again the PDF $f$ was determined a constant value $\mu_0$ and $\asD$ ($f=f(\mu_0,\asD)$).

Therefore, the two operators $\Gamma$ and $\gamma$ can be commonly expressed as:
\begin{align}
  \gamma^{(I)} :&= \gamma(\muf,\asA,\asD) \otimes \Gamma(\mu_0,\muf,\asD)\\
                &=  \rm(todo...~(help~needed)).
\end{align}




\section{Alternative}
\subsection{Formalism}
In order to enforce consistency of the values of \as\ in all components of the calculation, one may introduce
an arbitrary scale $\mu_0^\prime$ and write:
\begin{align}
  \sigma^{(II)} &= \sHat(\muf,\asA) \otimes \Gamma(\mu_0^\prime,\muf,\asA) \otimes f(\mu_0^\prime,\asD).
\label{eq:alternative}
\end{align}
Here, we take the PDF $f(\mu_0^\prime,\asD)$ from the PDF fitting groups, but not at the starting scale as they do, but
at another (higher) scale $\mu_0^\prime$. By doing so, we generate our own PDF $\tilde{f}(\muf,\asA)$, which is evolved with a value of $\asmz=\asA$ to all scales \muf.
By choosing the value of $\mu_0^\prime$ carefully, the new PDF $\tilde{f}(\muf,\asA) = \Gamma(\mu_0^\prime,\muf,\asA) \otimes f(\mu_0^\prime,\asD)$ 
will still be able to describe the bulk of the HERA inclusive DIS data.
A good value  of $\mu_0^\prime$ is expected to be around $20-30\,\GeV$, but depends of course
on the lever arm of low-$\mu$ and high-$\mu$ data, which is included in the PDF fit.
Such, the values of \asA\ and \asD\ can become identical, and full consistency is obtained.
Since no PDF-refitting is required, we do not introduce ambiguities that are introduced when 
refitting the PDFs with different values of \asD.

\subsection{Implementation}
The operator $\Gamma(\mu_0^\prime,\muf,\asA)$, can also be expressend in terms of the factor $\sqrt{(K^\prime)}$ as,
\begin{align}
  \log(K^\prime) = \rm... [help~needed]
\label{eq:kappaprime}
\end{align}
and therefore, we can take the evolution from the LHAPDF grids.

\subsection{Discussion}
When attempting to express equation~\ref{eq:alternative} in terms of the PDF-fitter's initial-scale PDF,
one re-writes equation~\ref{eq:alternative} as:
\begin{align}
  \sigma^{(II)} &= \sHat(\muf,\asA) \otimes \Gamma(\mu_0^\prime,\muf,\asA) \otimes \Gamma(\mu_0,\mu_0^\prime,\asD) \otimes f(\mu_0,\asD).
\end{align}
and we can define:
\begin{align}
  \gamma^{(II)} :&= \Gamma(\mu_0^\prime,\muf,\asA) \otimes \Gamma(\mu_0,\mu_0^\prime,\asD) \\
                 &= \rm(todo... (help~needed)).
\end{align}

We see, that the operators $\gamma^{(I)}$ and $\gamma^{(II)}$ differ by (or not...) [help~needed].



\section{General discussion}
It is now interesting to study the case, where one chooses the factorisation scale to be constant,
i.e.\ not as a `dynamical' scale choice but for instance $\muf=\mu_0^\prime$ (which may be for instance $\mu_0^\prime=20\,\GeV$).
In this case, the evolution operator disappears, and the cross section is simply expressed as 
initially requested by the factorisation theorem similar to eq.~\ref{eq:sig0} as:
\begin{align}
  \sigma^{(III)} &= \sHat(\mu_0^\prime,\asA) \otimes f(\mu_0^\prime,\asD).
  \label{eq:sigIII}
\end{align}
As a consequence, large logarithms appear, which are no longer absorbed into the PDF (you know better what happens...).

This raises the questions:
\begin{itemize}
  \item What exactly happens here, and what happens with the ``log's'' in the case of equation~\ref{eq:sigIII} in comparision to eq.~\ref{eq:alternative} ? Do we obtain a prediction formally of order NNLO+NNLL(ISR) ??
  \item In case of renormalisation, we also have to introduce a fixed value, and we have the running. Is there some analogy of the formalisms?
  \item Does this imply, that if we have a running coupling \asmur, i.e.\ we choose a `dynamic' renormalisation scale,
    at the same time we are obliged to introduce the evolution operator into the cross section definition?
  \item What happens, if we choose $\mur=\muf=mz$, where we don't have a running \asmur\ nor a `running' PDF?
\end{itemize}











\end{document}
