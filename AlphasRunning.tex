\documentclass[10pt,a4paper]{article}
\usepackage{amsmath,amssymb,bm,makeidx,subfigure}
\usepackage[italian,english]{babel}
\usepackage[center,small]{caption}[2007/01/07]
\usepackage{fancyhdr}
\usepackage{color}

\begin{document}

\section{Measuring the running of $\alpha_s$ from HERA DIS inclusive
  data}

In this set of notes we collect a number of ideas to measure the
running of the strong coupling $\alpha_s$ from the HERA I+II combined
DIS inclusive data. This will be carried out by a simultaneous fit of
PDFs to this data. Throughout these notes cross sections are assumed
to be renormalised and factorised in the $\overline{\mbox{MS}}$
scheme. We will focus on two main strategies that should lead to
consistent results as they are both based on QCD factorisation.

\subsection{Strategy one}

We start by observing that in fit of PDFs the strong coupling
$\alpha_s$ enters both the computation of the partonic cross sections,
which are perturbative quantities and thus are typically expressed as
truncated series in powers of $\alpha_s$, and the DGLAP evolution
where terms of the form $\alpha_s^n\ln^n(Q^2/Q_0^2)$ are resummed to
all orders. Focusing on DIS, the prediction for a certain cross
section in terms of initial scale PDFs is given by:
\begin{equation}\label{eq:preditions}
  \sigma(Q) = \hat{\sigma}(Q)\otimes\Gamma(Q,Q_0)\otimes f_0\,.
\end{equation}
In Eq.~(\ref{eq:preditions}) the partonic cross section $\hat{\sigma}$
allows for the perturbative expansion:
\begin{equation}\label{eq:exppartxsec}
  \hat{\sigma}(Q) = \hat{\sigma}_0 +\hat\alpha_s
  \hat{\sigma}_1 + \hat\alpha_s^2 \hat{\sigma}_2 + \dots
\end{equation}
where we have defined:
\begin{equation}
  \hat\alpha_s \equiv \alpha_s(Q)\quad\Longrightarrow \quad \hat{\sigma}(Q)
  \equiv \hat\sigma(\hat\alpha_s)\,.
\end{equation}
It should be noticed that the single perturbative terms
$\hat{\sigma}_i$ do depend explicitly on $Q$ through the ratios
$Q/\mu_R$, $Q/\mu_F$, and $Q/m_H$ but this is not relevant in the
following discussion because we are only looking at the dependence on
$\alpha_s$. The evolution kernel $\Gamma$ in Eq.~(\ref{eq:preditions})
is instead determined by solving the DGLAP equation:
\begin{equation}\label{eq:DGLAPEvOp}
Q^2\frac{d}{dQ^2}\Gamma(Q,Q_0) =
P(\alpha_s(Q))\otimes \Gamma(Q,Q_0)\,.
\end{equation}
with $P$ being the DGLAP splitting functions which are perturbatively
computable and are currently known to $\mathcal{O}(\alpha_s^3)$,
$i.e.$ NNLO accuracy. Using the ``obvious'' boundary condition:
\begin{equation}
\Gamma(Q_0,Q_0) = 1\,,
\end{equation}
one can schematically write the solution to Eq.~(\ref{eq:DGLAPEvOp})
as follows:
\begin{equation}\label{eq:EvolOp}
\Gamma(Q,Q_0) = \exp\left[\int_{\ln Q_0^2}^{\ln Q^2}P(\alpha_s(Q'))\,d\ln Q'^2\right]\,.
\end{equation}
It is clear from Eq.~(\ref{eq:EvolOp}) that the evolution operator
$\Gamma$ depends on all the values that $\alpha_s$ takes between $Q_0$
and $Q$. In other words, we need to know how $\alpha_s$ evolves in
this range. But this only depends on one single input that is the
value of $\alpha_s$ at some reference scale. Such value can be chosen
to be $\hat\alpha_s$ that thus means:
\begin{equation}
\Gamma\equiv \Gamma(\hat\alpha_s)\,,
\end{equation}
where we have omitted the dependence on $Q$ and $Q_0$ that we consider
fixed here. Finally, we notice that $f_0$ represents PDFs at the
initial scale. Since this quantity by definition does not undergo any
evolution, it only depends on the (fixed) initial scale $Q_0$ and not
on the value of $\alpha_s$ in $Q_0$. As a consequence, this quantity
can also be considered constant.

In the end of the day we can write Eq.~(\ref{eq:preditions}) as:
\begin{equation}\label{eq:preditions1}
  \sigma(Q) = \hat{\sigma}(\hat\alpha_s)\otimes\Gamma(\hat\alpha_s)\otimes f_0\equiv\sigma(\hat\alpha_s)\,,
\end{equation}
that is to say that the prediction for $\sigma(Q)$ only depends on
$\hat\alpha_s$.

Now, suppose to have a set of measurements corresponding to a number
of energies $Q_i$:
\begin{equation}
  \{\sigma_i = \sigma(Q_i) = \sigma(\hat\alpha_{s,i})\}\,.
\end{equation}
In principle, assuming that everything else is fixed ($i.e.$
perturbative order, PDFs, heavy quark masses, etc.) one can use each
of these measurements to determine $\hat\alpha_{s,i}$. This is
equivalent to determine the value of $\alpha_s$ for the different
values of the scale $Q_i$ which in turn means determining the running
of the strong coupling. There is a caveat though. In particular,
Eq.~(\ref{eq:EvolOp}) assumes that the running of the strong couplings
follows the RGE evolution. In fact, what we really aim at here is not
a direct measurement of the running of $\alpha_s$ but rather whether
the scaling violations of the measurements we are considering, which
are driven by the strong coupling, are consistent with the QCD
expectations.

As is well know, there is a close interplay between the value of
$\alpha_s$ and PDFs. Consequently, extracting the values of
$\hat\alpha_{s,i}$ keeping PDFs fixed represents an approximation that
might not be appropriate. To overcome this problem, one can
parametrize the initial scale PDFs $f_0$ by means of one of the usual
functional forms:
\begin{equation}
f_0\equiv f_0(\{a_j\})
\end{equation}
and fit the free parameters $a_j$ to data as commonly done in the
standard PDF determinations. As clear from Eq.~(\ref{eq:preditions1}),
$f_0$ is common to all predictions for $\sigma_i$ and thus one can no
longer determine the value of $\hat\alpha_{s,i}$ from $\sigma_i$ only
independently of the other measurements because now there is a
cross-talk between predictions due to $f_0$. Therefore, one needs to
perform a ``global fit'' to the set of measurements $\{\sigma_i\}$ and
the result of such a fit would be the determination of the two sets of
free parameters:
\begin{equation}\label{eq:freeparams}
\{\hat\alpha_{s,i}\}\quad\mbox{and}\quad\{a_j\}\,.
\end{equation}
This will (should) allow us to perform a simultaneous determination of
PDFs and the running of $\alpha_s$ avoiding the bias that a
determination of the running of the strong coupling obtained with
fixed PDFs might lead to.

It is interesting to notice that the approach described above
represents a generalisation of the standard procedure. As a matter of
fact, the reference values $\hat\alpha_{s,i}$ do not have to be
necessarily equal $\alpha_s(Q_i)$, but they can all be take to be
equal to the coupling to any arbitrary scale. The value of $\alpha_s$
at any scale $Q$ can then be obtained by evolution. In particular, one
can take all $\hat\alpha_{s,i}$ to be equal to one particular
value. This is what is typically done to determine the value of
$\alpha_s(M_Z)$, that is by simply setting:
\begin{equation}
\hat\alpha_{s,i} = \alpha_s(M_Z)
\end{equation}
for all measurements included in the fit. This clearly reduces the set
of free parameters to be determined the a global fit from
Eq.~(\ref{eq:freeparams}) to:
\begin{equation}
\alpha_s(M_Z)\quad\mbox{and}\quad\{a_j\}\,.
\end{equation}
While on the one hand this provides a simplification, this comes at
the price of assuming that the RGE running of the strong coupling is
used in a \textit{maximal} way also to completely determine the value
of the partonic cross section in Eq.~(\ref{eq:exppartxsec}) at the
relevant scale of the single measurements.

\end{document}
